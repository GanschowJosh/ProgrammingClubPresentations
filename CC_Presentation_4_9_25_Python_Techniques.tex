\documentclass{beamer}
\usetheme{Madrid}
\usepackage{listings}
\usepackage{xcolor}

\definecolor{codegray}{rgb}{0.95,0.95,0.95}
\lstset{
  backgroundcolor=\color{codegray},
  basicstyle=\ttfamily\small,
  frame=single,
  showstringspaces=false,
  language=Python,
  keywordstyle=\color{blue},
  commentstyle=\color{gray},
  stringstyle=\color{orange}
}

\title{Python Techniques That Will Make Your Professor\\Accuse You of Academic Dishonesty}
\author{Ethan Clark and Joshua Ganschow}
\date{\today}

\begin{document}

\frame{\titlepage}

\begin{frame}{Introduction}
Together, we're going to learn some secret techniques in Python that \textit{might} get you accused of academic dishonesty. We'll start off with some tame examples of what you can do, then get further into the weeds
\pause
\vspace{5em}

Disclaimer: We assume no responsibility if you use these and get accused.
\end{frame}

\section{Operator Overloading}
\begin{frame}[fragile]{Reinventing the Plus Operator}
\begin{lstlisting}
class Strange:
    def __init__(self, value):
        self.value = value
    def __add__(self, other):
        # Instead of summing, multiply the values
        return Strange(self.value * other.value)
        
a = Strange(3)
b = Strange(4)
print((a + b).value)  # Outputs 12 instead of 7
\end{lstlisting}
\end{frame}

\section{One-Liners That Shouldn't Work}
\begin{frame}[fragile]{One-Liner Examples}
\begin{lstlisting}
nums = [1, 2, 3, 4, 5]
squared_evens = [x*x for x in nums if x % 2 == 0]
\end{lstlisting}
\vspace{1em}
 
\begin{itemize}
  \item Creates a list of numbers.
  \item Uses a list comprehension to square each even number.
\end{itemize}
\vspace{1em}
\begin{lstlisting}
flat = [item for row in matrix for item in row]
\end{lstlisting}
\vspace{1em}
 
\begin{itemize}
  \item Flattens a 2D list (matrix) into a 1D list.
\end{itemize}
\end{frame}

\section{Dynamic Code Execution}
\begin{frame}[fragile]{Using \texttt{eval} and \texttt{exec}  }
\begin{lstlisting}
string_to_eval = "2*10"
eval(string_to_eval)
code = "print('Hello from exec')"
exec(code)
\end{lstlisting}

\end{frame}

\begin{frame}[fragile]{Recursive Lambda for Factorial Calculation}
\begin{lstlisting}[breaklines=true]
factorial = (
    lambda f: lambda x: 1 if x == 0 else x * f(f)(x - 1))
    (
    lambda f: lambda x: 1 if x == 0 else x * f(f)(x - 1)
)
print(factorial(5))
\end{lstlisting}
\vspace{1em}
\begin{itemize}
  \item The outer lambda accepts a function, and the inner lambda performs the recursion.
  \item The self-application trick (passing the lambda to itself) allows it to call itself recursively.
\end{itemize}
\end{frame}


\section{Decorators and Metaprogramming}
\begin{frame}[fragile]{Function Decorator}
\begin{lstlisting}
def evil_decorator(func):
    def wrapper(*args, **kwargs):
        print("Before execution")
        result = func(*args, **kwargs)
        print("After execution")
        return result
    return wrapper

@evil_decorator
def greet(name):
    print(f"Hello, {name}!")
    
greet("World")
\end{lstlisting}
\end{frame}

\begin{frame}[fragile]{Metaclasses for Adding Magic}
\begin{lstlisting}
class Meta(type):
    def __new__(self, name, bases, dct):
        dct['added_attribute'] = 'Mystery'
        return super().__new__(self, name, bases, dct)

class sub(metaclass=Meta):
    pass

print(sub.added_attribute)
\end{lstlisting}
\vspace{1em}
\textbf{When new is called}
\begin{itemize}
    \item \texttt{name} is the name of the class being created
    \item \texttt{bases} is a tuple of base classes
    \item \texttt{dct} is a ditionary containing the class's attributes and methods
\end{itemize}
\end{frame}

\section{Monkey Patching}
\begin{frame}[fragile]{Changing Behavior at Runtime}
\begin{lstlisting}
import math

original_sqrt = math.sqrt

math.sqrt = lambda x: 'Cheated sqrt!'

print(math.sqrt(16))

math.sqrt = original_sqrt
\end{lstlisting}
\vspace{1em}
 
\begin{itemize}
  \item Temporarily replaces the \texttt{math.sqrt} function with a lambda that returns a fixed string.
  \item Restores the original function afterwards.
\end{itemize}
\end{frame}

\section{Introspection with \texttt{inspect}}
\begin{frame}[fragile]{Inspecting Function Source Code}
\begin{lstlisting}
import inspect

def sample_function(a, b):
    return a + b

# Print the source code of sample_function
print(inspect.getsource(sample_function))
\end{lstlisting}
\vspace{1em}
\end{frame}

\section{Conclusion}
\begin{frame}{Closing Thoughts}
\begin{itemize}
  \item Python is powerful; apply it as such
  \item Maybe avoid some of these tricks if you can help it
\end{itemize}
\end{frame}

\end{document}
